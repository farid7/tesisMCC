%!TEX root = ../farid_msc_thesis.tex

\chapter{Introduction}
\label{ch:intro}

Business models can be understood as \textit{stories that explain how enterprises work} \cite{magretta2002business}, this histories have to take into account customers, what are the needs of customers and how this model is going to make money. But, there is also an extra factor that helps to improve the probability of successful of a business, and this factor is known as \textit{Design Innovation}.

\textit{Design Innovation}, or \textit{Experience Innovation}, are the ideas and methodologies for improving business models, encouraging innovation and growth \cite{liedtka2015perspective}. This approach suggest that an idea or business can achieve innovation if it is feasible (technology component), viable (can make money) and it is usable or desirable (human value of the idea). 

On the other hand, the number of proposal of business models has been increasing in the last years, and most of this ideas can be supported by \textit{startup founders}. Most of these startup founders are looking for great business ideas, providing them support and encouraging the development of new companies.

For this reason, we developed this work in order to create a methodology to identify interesting business models with high chances of success. This paper describes a proposed methodology that ranks business models applications, based on some \textit{Design Innovation} concepts \cite{liedtka2015perspective}. The rest of the paper is organized as follows. In Section 2 we briefly review some concepts of Natural Language Processing (NLP), Text Mining techniques and Fuzzy Inference Systems (FIS). In section 3, we describe the steps of our proposed methodology for ranking Business Models applications. In section 4 we present experimental results using real data. Finally, conclusions are discussed in Section 5.  

