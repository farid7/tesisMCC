% Tesis ITAM CLASS -- version 0.1 (13 - Abr - 2015)
% Clase para las tesis del ITAM
% 
% 13 - Abr - 2015 	Victor Martinez 	victor.martinez (at) itam.mx
% LICENSE: Creative Commons SA-BY 3.0
%
%
% Este documento presenta un ejemplo de uso de la plantilla
% El estudiante es libre de modificar este archivo a su gusto
% 
\documentclass{tesisITAM}
\usepackage[utf8]{inputenc}
\usepackage{amsfonts}
\usepackage{amsmath}
%\usepackage{algorithmx}
%\usepackage[spanish,onelanguage]{algorithm2e} 
%\RestyleAlgo{boxruled} % Para que los algoritmos los ponga en caja
\usepackage[backend=bibtex]{biblatex}
\usepackage[paperwidth=17cm, paperheight=22.5cm, bottom=2cm, left=2.5cm, right=2cm, headheight=13.08pt]{geometry}
%\usepackage[group-separator={,}]{siunitx}
\usepackage{numprint}
\usepackage{csquotes}
\usepackage{float}
\usepackage{multirow}
\usepackage{url}

\addbibresource{bibliography.bib} %Imports bibliography file

\title{Tesis de Farid}
\author{Angel Farid Fajardo Oroz}
\degree{Maestría en Ciencias de la Computación}
\advisor{Víctor Manuel González y González}
\year{2017}

\begin{document}
	\npthousandsep{,}
	\pagenumbering{gobble}
	\maketitle
	\publicationrights
	

	%!TEX root = ./tesis_mbc.tex
%\thispagestyle{empty}
\chapter*{\phantom{Dedicatoria}}

\hfil \textit{A mi gente.} \hfil




	%!TEX root = ./tesis_mbc.tex
%\thispagestyle{empty}
\chapter*{Agradecimientos}

A la otra gente.


	%%%%%%%%%%%%%%%%%%%%%%%%%%%%%%%%%%%%%%%%%%%%%%
	% ABSTRACT
	%%%%%%%%%%%%%%%%%%%%%%%%%%%%%%%%%%%%%%%%%%%%%%

	\begin{abstract}{spanish}
		Una tesis muy bonita.\\
		wiiiii!!!
	\end{abstract}

	\begin{abstract}{english}
The number of proposals of business model has been increasing in the last years, in that way that some organizations that give support to those \textit{}{start-ups} need to select the best ideas with high probability of success, among a wide variety of applications. For this reason, we propose a methodology for ranking business model applications using text mining techniques and a fuzzy rule based system. We evaluated this proposed methodology using real data applications from a company that evaluates those business proposals.

The applications we ranked consisted in questions and answers, but the business model description is written as the response to some open questions as "What is the need or problem you want to solve with your business idea?" or "What population are you benefiting/helping with your business idea?". For this reason, Natural Language Processing techniques had been applied in order to extract meaningful semantic features from texts, mapping some key-words from applications into a vectorial space. Then, we built an expert system whose rules were created taking into account the technological, social and economical impact of the business idea. Finally, the applications were sorted according to a given score that was assigned by our expert system.\\\dots

	\end{abstract}

	\selectlanguage{english}
	\setcounter{page}{1}
	\pagenumbering{roman}

	\tableofcontents
	\listoffigures
	\listoftables
	\newpage

	\pagenumbering{arabic}
	\setcounter{page}{1}

	%%%%%%%%%%%%%%%%%%%%%%%%%%%%%%%%%%%%%%%%%%%%%%
	% CONTENT
	%%%%%%%%%%%%%%%%%%%%%%%%%%%%%%%%%%%%%%%%%%%%%%

	%!TEX root = ../farid_msc_thesis.tex

\chapter{Introduction}
\label{ch:intro}

Business models can be understood as \textit{stories that explain how enterprises work} \cite{magretta2002business}, this histories have to take into account customers, what are the needs of customers and how this model is going to make money. But, there is also an extra factor that helps to improve the probability of successful of a business, and this factor is known as \textit{Design Innovation}.

\textit{Design Innovation}, or \textit{Experience Innovation}, are the ideas and methodologies for improving business models, encouraging innovation and growth \cite{liedtka2015perspective}. This approach suggest that an idea or business can achieve innovation if it is feasible (technology component), viable (can make money) and it is usable or desirable (human value of the idea). 

On the other hand, the number of proposal of business models has been increasing in the last years, and most of this ideas can be supported by \textit{startup founders}. Most of these startup founders are looking for great business ideas, providing them support and encouraging the development of new companies.

For this reason, we developed this work in order to create a methodology to identify interesting business models with high chances of success. This paper describes a proposed methodology that ranks business models applications, based on some \textit{Design Innovation} concepts \cite{liedtka2015perspective}. The rest of the paper is organized as follows. In Section 2 we briefly review some concepts of Natural Language Processing (NLP), Text Mining techniques and Fuzzy Inference Systems (FIS). In section 3, we describe the steps of our proposed methodology for ranking Business Models applications. In section 4 we present experimental results using real data. Finally, conclusions are discussed in Section 5.  



	%!TEX root = ../mbc_msc_thesis.tex

\chapter{Related work}
\label{ch:related_work}

lalala

	%!TEX root = ../mbc_msc_thesis.tex

\chapter{Data and methodology}
\label{ch:methodology}

lalala

	%!TEX root = ../mbc_msc_thesis.tex

\chapter{Results}
\label{ch:results}

lalalalala

	%!TEX root = ../farid_msc_thesis.tex

\chapter{Conclusions and future work}
\label{ch:conclusions}

lalala \ref{fig:MCC}


	

	%%%%%%%%%%%%%%%%%%%%%%%%%%%%%%%%%%%%%%%%%%%%%%
	% APPENDIX
	%%%%%%%%%%%%%%%%%%%%%%%%%%%%%%%%%%%%%%%%%%%%%%
	\appendix
	% \include{Chapters/appendixA}

	%%%%%%%%%%%%%%%%%%%%%%%%%%%%%%%%%%%%%%%%%%%%%%
	% BIBLIOGRAPHY
	%%%%%%%%%%%%%%%%%%%%%%%%%%%%%%%%%%%%%%%%%%%%%%
	\clearpage
	\addcontentsline{toc}{chapter}{References} 


% @misc{EncyMath_loss_func,
%   author = {{Encyclopedia of Mathematics}},
%   title = {Loss function},
%   url = {http://www.encyclopediaofmath.org/index.php?title=Loss_function&oldid=40954}
% }	
	
% @book{christopher2006pattern,
%   title={Pattern recognition and machine learning},
%   author={Christopher M.. Bishop},
%   year={2006},
%   publisher={Springer}
% }

% @book{nocedal2006numerical,
%   title={Numerical Optimization, Second Edition},
%   author={Nocedal, Jorge and Wright, Stephen J},
%   year={2006},
%   publisher={Springer New York}
% }

% @Manual{R_manual,
%   title = {R: A Language and Environment for Statistical Computing},
%   author = {{R Core Team}},
%   organization = {R Foundation for Statistical Computing},
%   address = {Vienna, Austria},
%   year = {2016},
%   url = {https://www.R-project.org/},
% }

% @Article{Rcpp_Article_Eddelbuettel_Francois,
%   title = {{Rcpp}: Seamless {R} and {C++} Integration},
%   author = {Dirk Eddelbuettel and Romain Fran\c{c}ois},
%   journal = {Journal of Statistical Software},
%   year = {2011},
%   volume = {40},
%   number = {8},
%   pages = {1--18},
%   url = {http://www.jstatsoft.org/v40/i08/},
% }

% @Book{Rcpp_Book_Eddelbuettel,
%   title = {Seamless {R} and {C++} Integration with {Rcpp}},
%   author = {Dirk Eddelbuettel},
%   publisher = {Springer},
%   address = {New York},
%   year = {2013},
%   note = {ISBN 978-1-4614-6867-7},
% }
 
% @Book{wickham_ggplot2,
%   author = {Hadley Wickham},
%   title = {ggplot2: Elegant Graphics for Data Analysis},
%   publisher = {Springer-Verlag New York},
%   year = {2009},
%   isbn = {978-0-387-98140-6},
%   url = {http://ggplot2.org},
% }
 
% @Manual{wickham_tidyverse,
%   title = {tidyverse: Easily Install and Load 'Tidyverse' Packages},
%   author = {Hadley Wickham},
%   year = {2016},
%   note = {R package version 1.0.0},
%   url = {https://CRAN.R-project.org/package=tidyverse},
% }




\begingroup
\raggedright
\sloppy
\printbibliography
\endgroup

\nocite{*}

\end{document}

%%% Local Variables:
%%% mode: latex
%%% TeX-master: t
%%% End:
